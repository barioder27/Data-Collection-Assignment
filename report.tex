\documentclass[options]{article}
\begin{document}
\title{A REPORT ON THE TRAFFIC JAM AROUND MAKERERE UNIVERSITY HILL}
\author{BARIYO DERRICK
15/U/4401/EVE
215008105}
\date{17/05/2017}
\maketitle






\subsection{\textbf{AUTHOR DETAILS}}
Bariyo Derrick 15/U/4401/EVE    215008105
\section{\textbf{Terms of Reference}}
A report submitted in fulfilment of the requirements of The Juicing Bible and
 http://www.wikihow.com/Make-Juice-Recipes

\section{\textbf{Summary}} 
Making Juice involves steps that transfer solid fruits into a juice which is mixed with a mixture of water
  and other ingredients that give a flavor and thick color to the to the Juice. Sugar is finally added to
 the mixture to give it a better taste. This is done using a blender.

\section{\textbf{Introduction}} 
In this report I talk about the different ingredients needed to make juice, the quantity of the ingredients
 and the steps taken to make the juice. 
I also state the different equipment that is needed when making juice.

\section{\textbf{Body}} 
A couple of steps are taken to make juice and all these are stated below. 
 \subsection{\textbf{These below are the required fruits;-}}  

Oranges 

Mangoes 

Pine apples 

Water melon 

Carrots 

Passion fruits 

The other ingredients needed include Sugar and cold water.

 \subsection{\textbf{The required equipment include;-}}  

A blender.

A sharp knife.

Clean containers to hold the juice. 

The following are the steps that are taken to make juice.

1.	Wash and rinse all the fruits, first, then remove any parts that might be tough or bitter.

2.	Remove the eyes from pineapples, and the fibrous core then slice them in to small pieces.

3.	Pill the mangoes, water melon and carrots and get the juice out of the passion fruits.

4.	Split the mangoes, carrots and watermelons and remove their seeds and any stems.

5.	Chop your fruits and vegetables into the appropriate-sized chunks.

6.	Put the sliced pieces of the fruits in to the blender and blend them all until they are fluid.

7.	Mix all the different juices that you have gotten so that you obtain your required flavor.

8.	Sieve the juice so as to filter out the fruit particles that are still solid.

9.	Add cold water to the juice according to the quantity of the juice that you had blended. 

The juice is ready and you can now enjoy your juice.

\section{\textbf{Conclusion }} 
Making juice is easy if the right ingredients and procedures are known. Make sure you never
 forget adding sugar to the juice, removing the seeds of the different fruits so that they are
 not blended and finally adding the right quantity of water to the juice.
\end{document}