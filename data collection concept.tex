\documentclass[options]{article}
\begin{document}
\title{A REPORT ON THE TRAFFIC JAM AROUND MAKERERE UNIVERSITY HILL}
\author{
BARIYO DERRICK
15/U/4401/EVE
215008105}
\date{17/05/2017}
\maketitle

\section{\textbf{ Introduction}}
Traffic jam is a condition on the road transport system that occurs as the number of cars on the road increases. This is usually characterized by slow speeds and vehicle queuing. 
\section{\textbf{ Background }}
The topic the research is about concerns the jam that is along the different rounds that are around Makerere hill.
The jam that occurs on these roads is brought about by many different factors that are discussed below 

These factors include:
\begin{itemize}
   \item The poor road structures of these roads, most of these roads only support two lanes hence limiting the number of cars that the road can accommodate a specific period.
   \item  Some of the roads are under construction which is a temporary factor which also causes the jam since cars blocked and forced to move at a slower sleep when the approach these areas for example the Makerere hill road which is under construction currently.
   \item Most of these roads have junction which bring about jam in a way that cars have to slow down and wait for the wright moment to go through the junction hence causing the jam 
\end{itemize}
This research focuses mostly on the roads that are around Makerere hill and these include Makerere hill road, Bombo road, Sir Apollo road and Gadafi road.

\section{\textbf{ Problem Statement}}
Traffic has been noticed along these rounds over past two weeks and it has been noticed that it can hold up vehicles for about ten to thirty minutes and this gets people late on the different programs that they need to follow and with some of these roads under constructions vehicles can be queued up for longer periods.

\section{\textbf{ Aim and Objectives}}
The aim of this project is to get solutions and ways through which the traffic jam along these roads can be reduced to lowest possible numbers.
Some ways of reducing this jam is through formulating ways through which people can get earlier updates about the routes that they plan to use so than they can plan otherwise for routes with lesser or no jam.
The jam can also be reduced by some people using so other alternative routes to their destinations.


\section {\textbf{Research Scope}}
This is a research on only the roads around the Makerere hill that is only the roads that are around Makerere University. That’s where all the research will be based and those will be the sources of all the data which is to be collected. 
The data to be collected includes videos and pictures of the jam on the different roads.
There is challenge of collecting data like pictures and videos in such a way that some few people don’t want to be included in pictured being taken by strangers.
There is also a challenge of getting the wright location coordinates since they are not 100 percent accurate.

\section {\textbf{Methodology}}

\begin{enumerate}
  \item \textbf{Nonperturbative renormalization group transformation (NRG),}


   \item \textbf{Quantum Spectral Transform Method (QSTM), }

\begin{itemize}
   \item 
   \item  The solution of the quantum inverse problem for the nonlinear.
\end{itemize}

   \item \textbf{Matrix product states and projected entangled pair states (MPS).}

\end{enumerate}

\section{\textbf{References}}
R.P.Feynman, “Simulating physics with computers,”

https://physics.aps.org/articles/v9/66 

W.P.schleich,Quantum Optics in Phase Space 



\end{document}